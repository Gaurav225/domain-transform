\section{Background: Whatever the Background is}\label{sec:background}
\fixme{Write.}

The original publication on the Domain Transform Filtering Technique ~\cite{GastalOliveira2011DomainTransform} gives  an in depth explanation of the theoretical foundations of the technique as well as presenting three different implementation approaches. These three approaches are called "Normalized convolution" (NC), "Interpolated convolution" (IC) and "Recursive filtering" (RF). 

In this section we will present the necessary background information to follow our implementation of the first of these three algorithms, namely "Normalized convolution". For a more detailed explanation of the method and it's background, please refer to the original publication.

The straight forward approach to image bluring is to compute a new value for each image pixel using the weighted average of its neighbourhood. To decide whether a pixel lies in this neighbourhood, we need some kind of distance measure. In the case of simple edge unaware bluring (such as applying a Gaussian filter), this distance measure is the spatial distance between image pixels. However to achieve edge-aware image filtering the radiometric distance between pixels (their difference in color) has to be taken into account as well.
In general this increases the dimensionality of the problem by adding a dimension for each color channel.

For a RGB color image this would lead to a 5-dimensional space (3 radiometric + 2 spatial dimensions). The domain transform method reduces this dimensionality as follows:
First, it treats the two spatial dimensions separately. What this means is that the image is separately filtered along its rows and columns. Secondly, it merges the three radiometric dimensions and the remaining one spatial dimension into the one-dimensional space of the domain transform. This transformation retains the necessary distance properties from the original space.


\mypar{Cost Analysis}
In asymptotic terms, the algorithm has a complexity of \BigO{n}, which makes it already very competitive with other approaches to edge-aware image filtering. For optimization purposes we are however interested in a more thorough cost analysis of the method. Therefore we have counted the number of different floating point operations executed by the algorithm:



\comment{
Give a short, self-contained summary of necessary
background information. For example, assume you present an
implementation of FFT algorithms. You could organize into DFT
definition, FFTs considered, and cost analysis. The goal of the
background section is to make the paper self-contained for an audience
as large as possible. As in every section
you start with a very brief overview of the section. Here it could be as follows: In this section 
we formally define the discrete Fourier transform, introduce the algorithms we use
and perform a cost analysis.

\mypar{Discrete Fourier Transform}
Precisely define the transform so I understand it even if I have never
seen it before.

\mypar{Fast Fourier Transforms}
Explain the algorithm you use.

\mypar{Cost Analysis}
First define you cost measure (what you count) and then compute the
cost. Ideally precisely, at least asymptotically. In the latter case you will need to instrument your code to count
the operations so you can create a performance plot.

Also state what is
known about the complexity (asymptotic usually) 
about your problem (including citations).

Don't talk about "the complexity of the algorithm.'' It's incorrect,
remember (Lecture 2)?
}