%\section{Your Proposed Method}\label{sec:method}
\section{Optimizing Domain Transform}
We started with the matlab code provided on the homepage\footnote{\url{http://www.inf.ufrgs.br/~eslgastal/DomainTransform/}} of the authors~\cite{GastalOliveira2011DomainTransform} and ported the code to C++. 
\subsection{Optimization Steps}
\subsubsection{Basic version}
In order to achieve the biggest optimisation potential the matrices have been wrapped in a simple struct (listing \ref{code:matrix_struct}).
\begin{lstlisting}[caption=Matrix struct,label=code:matrix_struct]
struct Mat2
{
    uint width, height;
    float3* data; // size: width*height
};
\end{lstlisting}
In order to preserve accuracy all images are stored in a \lstinline{float3} struct. Which uses $3\times 4 = 12$ bytes per pixel. Hence a $1$M pixel image is stored in $3\times 4\times 10^6$\ B $=12$\ MB.
The algorithm starts by precomputing $dIdx$ and $dIdy$ the differential of the image in both $x$ and $y$ direction. Then the prefix sum of both $dIdx$ and $dIdY^T$ is compute. 
During the filtering step we (listing \ref{code:filterstep}) compute \lstinline{computeBound} and \lstinline{boxFilter} twice. For each iteration step we transpose the images.
\begin{lstlisting}[caption=Filterstep,label=code:filterstep]
for (i=0; i<iterations; i++) {
    bounds = computeBound(img, dIdX, r);
    img = boxFilter(img, bounds);
    img = transpose(img);
    bounds = computeBound(img, dIdY, r);
    img = boxFilter(img, dIdY, bounds);
    img = transpose(img);
}
\end{lstlisting}
The \lstinline{computeBound} step computes the one dimensional bound for each pixel for the box filter step. The bound is dependant on a user defined radius $r$. 
During the box filter step the prefix sum of the image in the $x$ direction is precomputed with a separate function. During the box filter step the box filter makes use of the previously precomputed prefix sum and bounds to compute the filtered value for each pixel (listing \ref{code:boxfilter}).
\begin{lstlisting}[caption=Boxfilter step, label=code:boxfilter]
void boxfilter(img, bounds) {
    psum = prefixsum(img);
    for (i = 0; i < H; i++) {
        for (j = 0; j < W; j++) {
            l = lowerBound(i,j);
            r = upperBound(i,j);
            delta = r - l;
            img(i,j) = psum(l)-psum(r);
            img(i,j) = img(i,j)/delta;
        }
    }
}
\end{lstlisting}
\subsubsection{Blocked Transpose}
With benchmarking the initial version we recognised that the transpose function was responsible for most of our computation time. Hence we blocked the transpose (listing \ref{code:blocked_transpose}).
\begin{lstlisting}[caption=Transpose block, label=code:blocked_transpose]
for (yo=0; yo<Hmod; yo+=BLOCK)
  for (xo=0; xo<Wmod; xo+=BLOCK)
    for (uint y=yo; y<yo+BLOCK; y++)
      for (uint x=xo; x<xo+BLOCK; x++)
        uint idx = y*W + x;
        uint idxT = x*H + y;
        out.data[idxT] = in.data[idx];
    xo = Wmod;
    for (uint y=yo; y<yo+BLOCK; y++)
      for (uint x=xo; x<W; x++)
        uint idx = y*W + x;
        uint idxT = x*H + y;
        out.data[idxT] = in.data[idx];
// Handle bottom row and bottom rightmost block
\end{lstlisting}
\subsubsection{Inlining Bound Computation}
In order for compiler to optimise the the code we inlined the bound computation.
\begin{lstlisting}[caption=Inlining of the bound computation, label=code:inlining]
void boxfilter(img, dx, boxR) {
    psum = prefixsum(img);
    for (i=0; i<H; i++)
    {
        posL = 0;
        posR = 0;
        for (j=0; j<W; j++)
        {
            // Bound computation
            dtL = dx(i,j) - boxR;
            dtR = dx(i,j) + boxR;
            while (dx(i,posL) < dtL 
                    && posL < W-1)
                posL++;
            while (posR < W && 
                   dx(i,posR < dtR )
                posR++;
            // Boxfilter step
        }
    }
\end{lstlisting}
\subsubsection{Recombination}
The inlining allowed us to inline the prefix sum into the boxfilter bound computation step (listing \ref{code:recombination}).
\begin{lstlisting}[caption=Recombination, label=code:recombination]
    while (dx(i,posL) < dtL 
            && posL < W-1)
        sum -= imgRow[posL];
        posL++;

    while (posR < W && 
           dx(i,posR < dtR )
        sum  += imgRow[posR];
        posR++;
    // Boxfilter step
    invD = 1.0f /(posR-posL);
    img(i,j) = sum * invD;
\end{lstlisting}
\subsubsection{Write transpose}
By directly storing the data transposed with were able to increase performance and reduce the number of transpose functions needed to be computed. We note that 


%\\
%\fixme{Write.}
%\comment{
%Now comes the ``beef'' of the paper, where you explain what you
%did. Again, organize it in paragraphs with titles. As in every section
%you start with a very brief overview of the section.
%
%For this class, explain all the optimizations you performed. This mean, you first very briefly
%explain the baseline implementation, then go through locality and other optimizations, and finally SSE (every project will be slightly different of course). 
%Show or mention relevant analysis or assumptions. A few examples: 
%1) Profiling may lead you to optimize one part first; 
%2) bandwidth plus data transfer analysis may show that it is memory bound; 
%3) it may be too hard to implement the algorithm in full generality: make assumptions and state them 
%(e.g., we assume $n$ is divisible by 4; or, we consider only one type of input image); 
%4) explain how certain data accesses have poor locality. Generally, any type of analysis adds value to your work.
%
%As important as the final results is to show that you took a structured, organized approach to the optimization and that you explain why you did what you did.
%
%Mention and cite any external resources including library or other code.
%
%Good visuals or even brief code snippets to illustrate what you did are good. Pasting large amounts of code to fill the space is not good.
%}